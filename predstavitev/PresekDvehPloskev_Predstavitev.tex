\documentclass{beamer}
\mode<presentation>
{
  \usetheme{default}      % or try Darmstadt, Madrid, Warsaw, ...
  \usecolortheme{default} % or try albatross, beaver, crane, ...
  \usefonttheme{default}  % or try serif, structurebold, ...
  \setbeamertemplate{navigation symbols}{}
  \setbeamertemplate{caption}[numbered]
} 

\usepackage[slovene]{babel}
\usepackage{graphicx, amsmath, listings, amssymb, commath, float}
\usepackage[utf8]{inputenc}
\usepackage[section]{placeins}

\title{Presek dveh implicitno danih ploskev}
\author[Avtorji]{Aljaž Verlič, Blažka Blatnik, Lina Lumburovska, Luka Tavčer\\
Mentor: Damir Franetič}
\date{5. junij 2017}
\begin{document}

\begin{frame}
  \titlepage
\end{frame}

\begin{frame}{Opis problema}
   V $\mathbb{R}^3$ imamo podani dve poljubni implicitno dani ploskvi, opisanimi z enačbama $f_{1}(x)$ = $C_{1}$ in $f_{2}(x)$ = $C_{2}$.\\
   Naša naloga je poiskati krivuljo $K$, ki predstavlja presek teh dveh ploskev.\\
   \vspace{5mm}
   Presek ploskev je množica rešitev nelinearnega sistema enačb:
   \begin{center}
		$f_{1}(x)$ = $C_{1}$\\$f_{2}(x)$ = $C_{2}$
	\end{center}
\end{frame}

\begin{frame}{Opis modela}
  Sistem lahko gledamo tudi, kot enačbe nivojnic funkcij  $f_{1}$ in $f_{2}$, krivulja K pa je presek teh nivojnic. Gradienta funkcij sta tako v vsaki točki krivulje preseka, pravokotna nanjo. To opišemo:
    \begin{center}
        	$F(x) = \dfrac{(grad f_{1}(x))x (grad f_{2}(x))}{\|grad f_{1}(x))x (grad f_{2}(x))\|}$
    \end{center}
   in označimo $x = x(t)$ naravno parametrizacijo krivulje K. Ta x je rešitev avtonomnega sistema diferencialnih enačb:
   \begin{center}
		$\dot{x} = F(x)$
	\end{center}
\end{frame}

\begin{frame}{Opis metod}
 Za reševanje sistema $\dot{x} = F(x)$ lahko uporabimo katero izmed numeričnih metod za reševanje diferencialnih enačb:
 \begin{itemize}
    \item Eulerjeva metoda
    \begin{itemize}
        \item Bolj logična in preprosta
        \item Komulativna napaka z vsakim korakom narašča
    \end{itemize}
    \item Runge-Kutta 4. reda
    \begin{itemize}
        \item Bolj natančna
        \item Napaka precej manjša, kot pri Eulerjevi metodi
    \end{itemize}
 \end{itemize}
\end{frame}

\begin{frame}{Eulerjeva metoda}
 \begin{itemize}
    \item Na vsakem koraku naslednjo točko $(x_{i+1},y_{i+1})$ dobimo tako, da se za h (korak) premaknemo vzdolž tangente na rešitev $(x_{i},y_{i})$. 
    \item Točka $(x_{i+1},y_{i+1})$ leži na drugi partikularni rešitvi kot $(x_{i},y_{i})$.
    \item Napaka na vsakem koraku je reda $O(h^2)$.
 \end{itemize}
	\begin{block}{Geometrijsko:}
    	\begin{figure}
	    \centering
    	\includegraphics[width=0.55\textwidth]{Euler_method_geom}
\end{figure}
    \end{block}
\end{frame}

\begin{frame}{Eulerjeva metoda brez popravljanja}
	Opazimo, da je Eulerjeva metoda brez popravljanja približka "blizu" pravilni rešitvi, vendar se napaka z iteracijami povečuje. \\
	\begin{center}
		\includegraphics[scale=0.30]{eul1}
	\end{center}
\end{frame}

\begin{frame}{Eulerjeva metoda brez popravljanja}
	Napake, ki se seštevajo, so na večjem intervalu bolj opazne.\\
	\begin{center}
		\includegraphics[scale=0.30]{eul2}
	\end{center}	
\end{frame}

\begin{frame}{RK4 metoda}
	, ki kaže da je bolj natančna kot euler (napaka manjša) (mogoče primer navadn + primer z ogromnim korakom=vsaj malo napake?)
\end{frame}

\begin{frame}{Newtonova metoda za popravljanje približka}
 \begin{itemize}
    \item Z opisanima metodama za reševanje DE enačb, dobimo na vsakem koraku le približek (Posebej razvidno iz Eulerjeve metode).
    \item Približek želimo popraviti tako, da bo spet ležal na krivulji preseka.
 \end{itemize}
 Dobljeni približek y, želimo popraviti na nek x, ki bo ležal na preseku. Če zapišemo $F(y) \cdot x = F(y)\cdot y $, nam to predstavlja enačbo ravnine, ki je zelo bliz normalni ravnini na krivuljo K. Z Newtonovo metodo z začetnim približkom y rešimo sistem enačb:
  \begin{center}
		$f_{1}(x)$ = $C_{1}$\\$f_{2}(x)$ = $C_{2}$\\ $F(y) \cdot x = F(y)\cdot y $
	\end{center}
\end{frame}

\begin{frame}{Eulerjeva metoda s popravljanjem z Newtonovo metodo}
	Ko za popravljanje napake uporabimo Newtonovo metodo, dobimo pravilno rešitev.
	\includegraphics[scale=0.2]{eul2}
	\includegraphics[scale=0.2]{eul1}
	\includegraphics[scale=0.2]{eul3_newt}
	
\end{frame}

\begin{frame}{Potrebni pogoji in Jacobijeva matrika}
	Potreben pogoj za delovanje metod je, da sta funkciji $f_{1}$ in $f_{2}$ parcialno odvedljivi in da ima Jacobijeva matrika parcialnih odvodov poln rang 2. Za uspešno delovanje Newtonove metode moramo poiskati Jacobijevo matriko leve strani sistema nelinearnih enačb.
	
	\begin{center}
		JG = $\begin{bmatrix}
		grad(f_{1}) \\
		grad(f_{2}) \\
		grad(\vec{v} \cdot \vec{x}) \\
		\end{bmatrix}$
		oziroma
		JG = $\begin{bmatrix}
		grad(f_{1}) \\
		grad(f_{2}) \\
		\hspace{1mm}grad(\vec{v}\hspace{0.5mm}^\intercal) \\
		\end{bmatrix}$
	\end{center}
\end{frame}

\begin{frame}{Razlaga adaptivnega koraka? (utemeljitev implementacije in zakaj je potrebna)}

\end{frame}

\begin{frame}{Koncna analiza parov ploskev za vsak primer + slike}
	Delovanje našega programa lahko preverimo s programom, ki smo ga napisali v Octave-u. Kot vhodne parametre mu podamo obe implicitno podani funkciji $f_{1}$, $f_{2}$, $C1$, $C2$, $grad(f_{1})$, $grad(f_{2})$. Določimo tudi začetni približek $x_{0}$, začetno dolžino koraka in pa parameter, ki določa metodo delovanja (Euler/Runge-Kutta).
\end{frame}

\begin{frame}{Primer 1}
	Začnemo z preprostim primerom sfere in ravnine, podane z enačbama:\\
	
	\begin{itemize} 
		\item $f_{1}(x,y,z)$ = $x^2 + y^2 + z^2$ = 4
		\item $f_{2}(x,y,z)$ = $3x + 2y + z$ = 1	
	\end{itemize} 
	\includegraphics[scale=0.3]{primer1_1}
	\includegraphics[scale=0.3]{primer1_2}
\end{frame}

\begin{frame}{Primer 2}
	Tudi primer sfere in valja je relativno "lep"\\
	
	\begin{itemize}  
		\item $f_{1}(x,y,z)$ = $x^2 + y^2 + z^2$ = 4
		\item $f_{2}(x,y,z)$ = $x^2 + y^2$ = 1
	\end{itemize} 
	\includegraphics[scale=0.3]{primer2_1}
	\includegraphics[scale=0.3]{primer2_2}
\end{frame}

\begin{frame}{Primer 3}
	Stvari malce otežimo s sfero in $f_{2}$
	
	\begin{itemize}  
		\item $f_{1}(x,y,z)$ = $x^2 + y^2 + z^2$ = 4
		\item $f_{2}(x,y,z)$ = $y^4 + log(x^2 + 1)z^2 - 4$ = 1
	\end{itemize} 
	\includegraphics[scale=0.3]{primer3_1}
	\includegraphics[scale=0.3]{primer3_2}
\end{frame}

\begin{frame}{Primer 4}
	Za konec pa\\
	
	\begin{itemize}  
		\item $f_{1}(x,y,z)$ = $x^2 + cos(y)z^2 - 12$ = 4
		\item $f_{2}(x,y,z)$ = $y^4 + log(x^2 + 1)z^2 - 4$ = 1
	\end{itemize} 
	\includegraphics[scale=0.3]{primer4_1}
	\includegraphics[scale=0.3]{primer4_2}
\end{frame}
\begin{frame}{Primer 5}
	\begin{itemize}  
		\item $f_{1}(x,y,z)$ = $e^{(-x^{2}+1)}+y^{2}+z^{2}$ = 3
		\item $f_{2}(x,y,z)$ = $e^{(xyz)}+y^{2}+z^{2}$ = 10
	\end{itemize} 
	\includegraphics[scale=0.3]{primer5_1}
	\includegraphics[scale=0.3]{primer5_2}
	\includegraphics[scale=0.3]{primer5_4} 
\end{frame}
\begin{frame}{Primer 6}
	\begin{itemize}  
		\item $f_{1}(x,y,z)$ = $e^{(-x^{2}+1)}+y^{2}+z^{2}$ = 3
		\item $f_{2}(x,y,z)$ = $x^2 + y^2 + z^2$ = 4
	\end{itemize} 
	\includegraphics[scale=0.3]{primer6_1}
	\includegraphics[scale=0.3]{primer6_2}
	\includegraphics[scale=0.3]{primer6_3} 
\end{frame}
\begin{frame}{Primer 7}
	\begin{itemize}  
		\item $f_{1}(x,y,z)$ = $e^{(-x^{2}+1)}+y^{2}+z^{2}$ = 3
		\item $f_{2}(x,y,z)$ =  $x^2 + y^2$ = 1
	\end{itemize} 
	\includegraphics[scale=0.3]{primer7_1}
	\includegraphics[scale=0.3]{primer7_2}
	\includegraphics[scale=0.3]{primer7_3} 
\end{frame}
\begin{frame}{Primer 8}
	\begin{itemize}  
		\item $f_{2}(x,y,z)$ = $e^{(xyz)}+y^{2}+z^{2}$ = 10
		\item $f_{2}(x,y,z)$ =  $x^2 + y^2$ = 1
	\end{itemize} 
	\includegraphics[scale=0.3]{primer8_1}
	\includegraphics[scale=0.3]{primer8_2}
	\includegraphics[scale=0.3]{primer8_3} 
\end{frame}

\end{document}
