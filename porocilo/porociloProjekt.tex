\documentclass[12pt]{article}
\usepackage[hidelinks]{hyperref}
\usepackage{graphicx, amsmath, listings, amssymb, commath}
\usepackage[utf8]{inputenc}
\usepackage[slovene]{babel}
\usepackage{amsmath}
\usepackage{enumitem}
\usepackage[utf8]{inputenc}
\usepackage{graphicx}
\usepackage{listingsutf8}
\usepackage{float}
\setlist[enumerate]{font=\large\bfseries}
\setlist[itemize]{font=\normalsize\bfseries}
\lstdefinestyle{mystyle}{
	showtabs=false, 
	tabsize=2
}

\lstdefinestyle{mystyle}{
	showtabs=false, 
	tabsize=2
}
\lstset{style=mystyle}
\usepackage[utf8]{inputenc}

\title{\small{2. projekt pri predmetu MATEMATIČNO MODELIRANJE} \\ \hfill \\ \hfill \\ \huge{\textbf{Presek dveh implicitno danih ploskev}}}
\author{Aljaž Verlič, Lina Lumborovska, Blažka Blatnik, Luka Tavčer \\
	Mentor: Damir Franetič}
\date{5. junij, 2017}
\begin{document}

\maketitle

\textbf{\large{Vsebina}} 
\begin{enumerate}
	\item Predstavitev problema
	\item Opis modela in uporabljenih metod
	\item Potrebni pogoj in Jacobijeva matrika
	\item Adaptivni korak
	\item Implementacija, testiranje in primeri
	\item Analiza za povpre\v{c}no \v{s}tevilo korakov Newtonove metode
	\item Koda
	\item Delitev dela v skupini
	\item Reference
\end{enumerate} 
\newpage

\section{Predstavitev problema}
	V $\mathbb{R}^3$ imamo podani dve poljubni implicitno dani ploskvi, opisani z enačbama $f_{1}(x)$ = $C_{1}$ in $f_{2}(x)$ = $C_{2}$. Presek teh dveh ploskev je množica rešitev nelinearnega sistema enačb:
	\begin{center}
		$f_{1}(x)$ = $C_{1}$,\\$f_{2}(x)$ = $C_{2}$.
	\end{center}
	Naša naloga je poiskati krivuljo $K$ (oz. bolj natan\v{c}no to\v{c}ke na njej), ki predstavlja presek teh dveh ploskev. \\

\section{Opis modela in uporabljenih metod}
    Sistem lahko gledamo tudi, kot enačbe nivojnic funkcij  $f_{1}$ in $f_{2}$, krivulja K pa je presek teh nivojnic. Gradienta funkcij sta tako v vsaki točki krivulje preseka, pravokotna nanjo. To opišemo:
    \begin{center}
        	$F(x) = \dfrac{(grad f_{1}(x))\times(grad f_{2}(x))}{\|grad f_{1}(x))\times(grad f_{2}(x))\|}$
    \end{center}
   in označimo $x = x(t)$ naravno parametrizacijo krivulje K. Ta x je rešitev avtonomnega sistema diferencialnih enačb:
   \begin{center}
		$\dot{x} = F(x)$
	\end{center}
	
\subsection{Re\v{s}evanje sistema diferencialnih ena\v{c}b}
    Za re\v{s}evanje sistema diferencialnih ena\v{c}b, lahko uporabimo katero izmed dveh znanih metod za numeri\v{c}no re\v{s}evanje diferencialnih ena\v{c}b.
\subsubsection{Eulerjeva metoda}
    Eulerjeva metoda je numeri\v{c}na metoda za re\v{s}evanje diferencialnih ena\v{c}b, z podanim za\v{c}etnim pribli\v{z}kom. Prednost metode je, da je preprosta in najbol logi\v{c}na. Na vsakem koraku naslednjo to\v{c}ko $(x_{i+1},y_{i+1})$ dobimo tako, da se za h (korak) premaknemo vzdol\v{z} tangente na re\v{s}itev $(x_{i},y_{i})$. Geometrijsko si lahko delovanje metode predstavimo s spodnjo sliko.
    
    \begin{figure}[H]
	    \centering
    	\includegraphics[width=0.5\textwidth]{Euler_method_geom}
    	\caption{Geometrijski prikaz delovanja Eulerjeve     metode $r$.}
    	\label{slika:Euler_method_geom}
	\end{figure}

	Slabost opisanega postopka je gotovo napaka, ki se skozi iteracije izvajanja metode pove\v{c}uje (napaka na vsakem koraku metode je reda $O(h^2)$, komulativna napaka pa z vsako iteracijo nara\v{s}\v{c}a). Tako je napaka ve\v{c}ja, tem ve\v{c}ji je korak. V iskanju re\v{s}itve na\v{s}ega problema to predstavlja te\v{z}avo, zato je potrebno sproti nara\v{c}unane pribli\v{z}ke vedno popraviti tako, da spet le\v{z}ijo na krivulji preseka. 
	Iz preprostega primera iskanja presečišča valja in sfere lahko opazimo delovanje Eulerjeve metode in problem komulativne napake: (Valja na sliki zaradi ve\v{c}je preglednosti ni)
	
	\begin{figure}[H]
	    \centering
    	\includegraphics[scale=0.30]{eul1}
    	\caption{Eulerjeva metoda pri manjsem koraku}
    	\label{slika:eul1}
	\end{figure}
	\begin{figure}[H]
	    \centering
    	\includegraphics[scale=0.30]{eul2}
    	\caption{Eulerjeva metoda pri večjem koraku}
    	\label{slika:eul2}
	\end{figure}

	
\subsubsection{Metoda Runge-Kutta 4}
    Precej bol natan\v{c}na, a manj intuitivna, metoda za re\v{s}evanje diferencialnih ena\v{c}b je metoda Runge-Kutta 4. Tudi tu potrebujemo nek za\v{c}etni pribli\v{z}ek, metoda pa potem z ve\v{c}jo natan\v{c}nostjo ra\v{c}una nadaljne premike. Napaka na vsakem koraku je enaka Eulerjevi metodi, vendar je komulativna napaka konstantna in se z iteracijami ne povečuje.

\subsubsection{Newtnova metoda za popravljanje pribli\v{z}ka}
    Kot je mo\v{c} opaziti, je pri uporabi Eulerjeve metode potrebno sprotno popravljanje pribli\v{z}kov, druga\v{c}e se napaka se\v{s}teva do te mere, da ne dobimo \v{z}eljene re\v{s}itve.\\
    Dobljeni pribli\v{z}ek y, \v{z}elimo popraviti na nek x, ki bo le\v{z}al na preseku. \v{C}e zapi\v{s}emo $F(y) \cdot x = F(y)\cdot y $, nam to predstavlja ena\v{c}bo ravnine, ki je zelo bliz normalni ravnini na krivuljo K. Z Newtonovo metodo z za\v{c}etnim pribli\v{z}kom y re\v{s}imo sistem ena\v{c}b:
    \begin{center}
    	$f_{1}(x)$ = $C_{1}$\\$f_{2}(x)$ = $C_{2}$\\ $F(y) \cdot x = F(y)\cdot y $
    \end{center}
    Re\v{s}itev sistema je to\v{c}ka, ki le\v{z}i na prese\v{c}i\v{s}\v{c}u obeh ploskev. V primeru uporabe RK4 je Newtnova metoda potrebna bolj za postavitev zacetnega pribli\v{z}ka na prese\v{c}i\v{sc}e, saj je metoda RK4 sama po sebi precej natan\v{c}na. U\v{c}inkovitost metode je razvidna iz primerov:
    
    \begin{figure}[H]
        \centering
        \includegraphics[scale=0.3]{eul1}
	    \includegraphics[scale=0.3]{eul3_newt}
	    \caption{Osnovna Eulerjeva metoda (levo) in popravljena Eulerjeva metoda (desno).}
    	\label{slika:eul1,eul3_newt}
	\end{figure}
    \begin{figure}[H]
        \centering
        \includegraphics[scale=0.30]{rk4}
    	\includegraphics[scale=0.30]{rk4_newt}
	    \caption{Osnovna metoda RK4 (levo) in popravljena metoda RK4 (desno).}
    	\label{slika:rk4,rk4_newt}
	\end{figure}

\newpage	
\section{Potrebni pogoj in Jacobijeva matrika}
	Potreben pogoj za delovanje metod je, da sta funkciji $f_{1}$ in $f_{2}$ parcialno odvedljivi in da ima Jacobijeva matrika parcialnih odvodov poln rang 2. Za uspešno delovanje Newtonove metode moramo poiskati Jacobijevo matriko leve strani sistema nelinearnih enačb:

	\begin{center}
		JG = $\begin{bmatrix}
		grad(f_{1}) \\
		grad(f_{2}) \\
		\hspace{1mm}grad(\vec{v} \cdot \vec{x}) \\
		\end{bmatrix}$
		oziroma
		JG = $\begin{bmatrix}
		grad(f_{1}) \\
		grad(f_{2}) \\
		\hspace{1mm}grad(\vec{v}\hspace{0.5mm}^\intercal) \\
		\end{bmatrix}$
	\end{center}
	Newtnova metoda rešuje sistem:
	\begin{center}
	    $\vec{F}(\vec{x}) = \vec{0}$,
	\end{center}
	zato je potrebno poiskati tudi matriko $\vec{F}(\vec{x})$, ki je enaka:
		\begin{center}
		$\vec{F}(\vec{x})$ = $\begin{bmatrix}
		f_{1}(\vec{v}) - C_{1} \\
		f_{2}(\vec{v}) - C_{2} \\
		\hspace{1mm}\vec{v} \cdot \vec{x} - \vec{v} \cdot \vec{y}) \\
		\end{bmatrix}$
	\end{center}

\section{Adaptivni korak}
		
\section{Implementacija, testiranje in primeri}
	Delovanje našega programa lahko preverimo s programom, ki smo ga napisali v Octave-u. Kot vhodne parametre mu podamo obe implicitno podani funkciji $f_{1}$, $f_{2}$, $C1$, $C2$, $grad(f_{1})$, $grad(f_{2})$. Določimo tudi začetni približek $x_{0}$, začetno dolžino koraka in pa parameter, ki določa metodo delovanja (Euler/Runge-Kutta).\\
	Program poženemo na različnih primerih in štejemo povprečno dolžino koraka ter število porabljenih korakov:\\\\
		
	\begin{minipage}{\textwidth}
	\textbf{\large{Primer 1:}}
	\begin{itemize} 
		\item $f_{1}(x,y,z)$ = $x^2 + y^2 + z^2$ = 4
		\item $f_{2}(x,y,z)$ = $3x + 2y + z$ = 1	
	\end{itemize}
	\begin{figure}[H]
	    \centering
    	\includegraphics[scale=0.3]{primer1_1}
    	\includegraphics[scale=0.3]{primer1_2}
    \end{figure}
    \end{minipage}
    
	\begin{minipage}{\textwidth}
	\textbf{\large{Primer 2:}}
	\begin{itemize}  
		\item $f_{1}(x,y,z)$ = $x^2 + y^2 + z^2$ = 4
		\item $f_{2}(x,y,z)$ = $x^2 + y^2$ = 1
	\end{itemize}
	\begin{figure}[H]
	    \centering
	    \includegraphics[scale=0.3]{primer2_1}
    	\includegraphics[scale=0.3]{primer2_2}
	\end{figure}
	\end{minipage}
	
		\begin{minipage}{\textwidth}
	\textbf{\large{Primer 3:}}
	\begin{itemize}  
		\item $f_{1}(x,y,z)$ = $x^2 + y^2 + z^2$ = 4
		\item $f_{2}(x,y,z)$ = $y^4 + log(x^2 + 1)z^2 - 4$ = 1
	\end{itemize}
	\begin{figure}[H]
	    \centering
	    \includegraphics[scale=0.3]{primer3_1}
    	\includegraphics[scale=0.3]{primer3_2}
	\end{figure}
	\end{minipage}
	
	\begin{minipage}{\textwidth}
	\textbf{\large{Primer 4:}}
	\begin{itemize}  
		\item $f_{1}(x,y,z)$ = $x^2 + cos(y)z^2 - 12$ = 4
		\item $f_{2}(x,y,z)$ = $y^4 + log(x^2 + 1)z^2 - 4$ = 1
	\end{itemize}
	\begin{figure}[H]
	    \centering
    	\includegraphics[scale=0.3]{primer4_1}
    	\includegraphics[scale=0.3]{primer4_2}
    	\includegraphics[scale=0.3]{primer4_4}
    	\includegraphics[scale=0.3]{primer4_3}
	\end{figure}
	\end{minipage}
	
	\begin{minipage}{\textwidth}
	\textbf{\large{Primer 5:}}
	\begin{itemize}  
		\item $f_{1}(x,y,z)$ = $e^{(-x^{2}+1)}+y^{2}+z^{2}$ = 3
		\item $f_{2}(x,y,z)$ = $e^{(xyz)}+y^{2}+z^{2}$ = 10
	\end{itemize}
	\begin{figure}[H]
    	\centering
    	\includegraphics[scale=0.4]{primer5_1}
    	\includegraphics[scale=0.4]{primer5_2} 
    	\includegraphics[scale=0.4]{primer5_3}
    	\includegraphics[scale=0.4]{primer5_4} 
	\end{figure}
	\end{minipage}
	
	\begin{minipage}{\textwidth}
	\textbf{\large{Primer 6:}}
	\begin{itemize}  
		\item $f_{1}(x,y,z)$ = $e^{(-x^{2}+1)}+y^{2}+z^{2}$ = 3
		\item $f_{2}(x,y,z)$ = $x^2 + y^2 + z^2$ = 4
	\end{itemize}
	\begin{figure}[H]
    	\centering
	    \includegraphics[scale=0.5]{primer6_1}
	    \includegraphics[scale=0.5]{primer6_2}
    	\includegraphics[scale=0.5]{primer6_3}
	    \includegraphics[scale=0.5]{primer6_4} 
	\end{figure}
	\end{minipage}
	
	\begin{minipage}{\textwidth}
	\textbf{\large{Primer 7:}}
	\begin{itemize}  
		\item $f_{1}(x,y,z)$ = $e^{(-x^{2}+1)}+y^{2}+z^{2}$ = 3
		\item $f_{2}(x,y,z)$ =  $x^2 + y^2$ = 1
	\end{itemize}
	\begin{figure}[H]
    	\centering
    	\includegraphics[scale=0.4]{primer7_1}
	    \includegraphics[scale=0.4]{primer7_2}
	    \includegraphics[scale=0.4]{primer7_3}
    	\includegraphics[scale=0.4]{primer7_4} 
    \end{figure}    
    \end{minipage}
    
	\begin{minipage}{\textwidth}
	\textbf{\large{Primer 8:}}
	\begin{itemize}  
		\item $f_{2}(x,y,z)$ = $e^{(xyz)}+y^{2}+z^{2}$ = 10
		\item $f_{2}(x,y,z)$ =  $x^2 + y^2$ = 1
	\end{itemize}
	\begin{figure}[H]
    	\centering
    	\includegraphics[scale=0.5]{primer8_1}
    	\includegraphics[scale=0.5]{primer8_2}
    	\includegraphics[scale=0.5]{primer8_3}
    	\includegraphics[scale=0.5]{primer8_4} 
    \end{figure} 
    \end{minipage}
\newpage
\section{Analiza za povpre\v{c}no \v{s}tevilo korakov Newtonove metode} 
Na koncu smo za vseh 8 primerov naredili analizo, tako da smo izmerili povpre\v{c}no \v{s}tevilo korakov Newtonove metode za RK4 in Eulerjevo matodo na dva na\v{c}ina: z adaptivnem in fiksnem korakom. Pri adaptivnem koraku smo izra\v{c}unali tudi minimalno in maksimalno \v{s}tevilo korakov. \newline
Primer in rezultati izvajanje enega testa: \\
	\includegraphics[scale=0.8]{a1}
	\includegraphics[scale=0.8]{a2}
	\includegraphics[scale=0.8]{a3}
	\includegraphics[scale=0.8]{a4} \\ \\
Po izvajanje tega postopka za vse primere smo dobili naslenje rezultate: \\
\begin{tabular}{|c | c | c | c | c |} 
 \hline
 \textbf{Funkcija} &\multicolumn{2}{|l|}{\textbf{Euler}} &\multicolumn{2}{|l|}{\textbf{RK4}}\\                
  \small &adaptivno & fiksno & adaptivno & fiksno \\ 
 \hline
 \small{}$f_{1}(x,y,z)=x^{2} + y^{2}+ z^{2}=4$ & & & &\\ 
 \small{}$f_{2}(x,y,z)=x^{2} + y^{2}    =1$  & 3 & 5 & 3 & 3.222 \\
 \hline
 \small{}$f_{1}(x,y,z)=x^{2} + y^{2}+ z^{2}=4$ & & & &\\ 
 \small{}$f_{2}(x,y,z)$ = $3x + 2y + z = 1$  & 3 & 4.222 & 3 & 3.333 \\
 \hline
 \small{}$f_{1}(x,y,z)=x^{2} + y^{2}+ z^{2}=4$ & & & &\\ 
 \small{}$f_{2}(x,y,z)$ = $y^4 + log(x^2 + 1)z^2 - 4 = 1$  & 3 & 5 & 3 & 3.222 \\
 \hline
 \small{}$f_{1}(x,y,z)$ = $x^2 + cos(y)z^2 - 12 = 4$ & & & &\\ 
\small{}$f_{2}(x,y,z)$ = $y^4 + log(x^2 + 1)z^2 - 4 = 1$  & 3 & 4.111 & 2.222 & 2.556 \\
 \hline
 \small{}$f_{1}(x,y,z)$ = $e^{(-x^{2}+1)}+y^{2}+z^{2} = 3$ & & & &\\ 
 \small{}$f_{2}(x,y,z)$ = $e^{(xyz)}+y^{2}+z^{2} = 10$  & 3 & 4.667 & 2.222 & 2.667 \\
 \hline
 \small{}$f_{1}(x,y,z)$ = $e^{(-x^{2}+1)}+y^{2}+z^{2} = 3$ & & & &\\ 
 \small{}$f_{2}(x,y,z)=x^{2} + y^{2}+ z^{2}=4$  & 3 & 5 & 3 & 3.222 \\
 \hline
 \small{}$f_{1}(x,y,z)$ = $e^{(-x^{2}+1)}+y^{2}+z^{2} = 3$ & & & &\\ 
  \small{}$f_{2}(x,y,z)=x^{2} + y^{2}    =1$  & 3 & 5 & 3 & 3.333 \\
  \hline
\small{}$f_{1}(x,y,z)$ = $e^{(xyz)}+y^{2}+z^{2} = 10$ & & & &\\ 
 \small{}$f_{2}(x,y,z)=x^{2} + y^{2}    =1$  & 3 & 5 & 3 & 3.556 \\
  \hline
\end{tabular} \par

Pri bolj kompleksnih funkcijah kot: $f_{1}(x,y,z)$ = $x^2 + cos(y)z^2 - 12 = 4$  in $f_{2}(x,y,z)$ = $y^4 + log(x^2 + 1)z^2 - 4 = 1$, pri adaptivnem koraku, \v{s}e posebej pri Runge-Kutta metodi, smo dobili manj\v{s}e povpre\v{c}no \v{s}tevilo korakov ker je metoda hitro konvergirala; pa tudi minimalno in maksimalno \v{s}tevilo korakov so bili manj\v{s}i kot pri vseh ostalih.\par
Glede na fiksnem koraku spet smo pri\v{s}li do istem zaklju\v{c}ku kot pri adaptivnemu, da bolj kompleksne funkcije hitrej\v{s}e konvergirajo in rabijo manj \v{s}tevilo korakov, ampak \v{s}e vedno ve\v{c} v primerjavi z adaptivnemu. 

\section{Koda}
\begin{lstlisting}[language=Octave]

\end{lstlisting}
\newpage
\section{Delitev dela v skupini}
\subsection{Programerski del}
\begin{itemize}
	\item Aljaž: Adaptivni korak, test skripta
	\item Lina: Re\v{s}evanje primerov, empiri\v{c}no dolocanje parametrov
	\item Blažka: Ogrodje programa, zdru\v{z}itev funkcij, test skripta
	\item Luka: Re\v{s}evanje primerov, empiri\v{c}no dolocanje parametrov
\end{itemize}
\subsection{Poro\v{c}ilo}
\begin{itemize}
	\item Aljaž: Adaptivni korak, Jacobijeva matrika
	\item Lina: Analiza, primeri, testiranje
	\item Blažka: Predstavitev problema, opis modela in metod
	\item Luka: Opis modela in metod (primeri delovanja)
\end{itemize}
\section{Reference}
\begin{itemize}
	\item Zapiski s predavanj: Diferencialne ena\v{c}be
	\item \url{https://en.wikipedia.org/wiki/Euler_method}
\end{itemize}
\end{document}
